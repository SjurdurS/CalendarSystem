\begin{versionhistory}
  \vhEntry{1.0}{21.10.14}{anud|ntho|\mbox{sjri}}{Revision: Removed non-functional requirements from Design goals. Updated component diagrams. More coherent.}
  \vhEntry{1.01}{22.10.14}{ntho|sjri}{Revision: Added more design goals: Low operating cost, Scalability, High Availability}
  \vhEntry{1.02}{22.10.14}{ntho|sjri}{Revision: Updated the access control matrix: Removed the objects Global, Local and Account and replaced them with Entity objects found in the RAD document. \newline Renamed the Client actors to User as it seemed more intuitive}
  \vhEntry{1.03}{22.10.14}{anud}{Revision: Updated Subsystem decomposition, Hardware/Software mapping and Persistent data management: Coherent subsystem names. Removed \textit{syncManagement} and \textit{shareManagement}. Removed Component Diagram 0.3 and replaced Component Diagram in subsystem decomposition with Component Diagram 0.4}
  \vhEntry{1.04}{23.10.14}{ntho, sjri, anud}{Revision: Updated access control text to the new matrix.
  \newline Renamed subsystem \textit{Calendar application} to \textit{Calendar Management}
  \newline Updated text in Boundary conditions.}
  \vhEntry{1.05}{28.10.14}{ntho|anud}{Updated UML Class diagram description.
  \newline Updated UML Class diagram.
  \newline Updated Component diagram.
  \newline Updated subsystem decomposition description.
  \newline Updated Hardware/software mapping description.
  \newline Updated Global software control description.
  \newline Updated subsystem services description.}
  \vhEntry{1.1}{29.10.14}{ntho}{Cleanup and reformatting.}
  \vhEntry{1.11}{01.11.14}{ntho, sjri, anud}{Updated UML Class Diagram.
  Added proxy design pattern to Calendar.
  Added an implementation of the IShare interface.
  Added Invitation class to the diagram.}
  \vhEntry{1.12}{02.11.14}{sjri}{Removed the Message class and added a Message string attribute to the Alarm class instead. Updated the UML Class Diagram and the architectural prototype accordingly.}
  \vhEntry{1.13}{26.11.14}{anud,sjri,ntho}{Updated class diagram}

\end{versionhistory}
\section{Access control and security}

The Calendar Application is a multiple user system, different actors are allowed to view different sets of objects, and invoke different types of operations on them. To document the different users access rights, we made an access control matrix (figure \ref{fig:AccessControlMatrix}). 
In summary \(user\) can create \(calendars\), \(evants\) and \(alarms\). Beside that the user have the ability to edit the \(calendar\), \textit{events} and \textit{alarm}. The server is only used to sychronize the \textit{calendar}. The \textit{other user}, is other users that locally on thier own \textit{calendars} have the same abilities as the \textit{user}. In this case the \textit{other user} only have the ability to subcribe, if the \(other user\) got an invite for the specific \textit{calendar}. And if not given, the other user can ask for permision to edit the \textit{calendar} or \textit{events}. \(Other user\), can of course also unsubsrcibe from the \textit{calendar} or \textit{event}.
\newline

\begin{figure}[hbp]
\hspace*{-1.5cm}\begin{tabular}{|p{1.8cm}|p{1.4cm}|p{3.7cm}|p{3.7cm}|p{2.4cm}|p{2.6cm}|}
\hline
\textbf{\space\space\space \mbox{Object}\newline Actors}
&
\textbf{Client}
&
\textbf{Calendar}
&
\textbf{Event}
&
\textbf{\mbox{Subscribing} User}
&
\textbf{Alarm} \\
\hline
\textbf{User}
& 
edit() \newline
delete() \newline
& 
getCalender()  \newline
setCalender()  \newline
subscribe() \newline
unsubscribe() \newline
getSubscribedUsers() \newline
save() \newline
delete() \newline
&
getEvent()  \newline
setEvent() \newline
subscribe() \newline
unsubscribe() \newline
getSubscribedUsers() \newline
save() \newline
delete() \newline
&
&
getAlarm()  \newline
setAlarm() \newline
removeAlarm() \newline
 \\ \hline
\textbf{Server}
& 
& 
synchronize() \newline
&
&
&
 \\ \hline
\textbf{Other User}
& 
& 
requestEditAccess()  \newline
subscribe()   \newline
unsubscribe() \newline
& 
requestEditAccess()  \newline
subscribe()  \newline
unsubscribe() \newline
& 
& 
 \\ \hline
\end{tabular}
\caption{\label{fig:AccessControlMatrix}Access control matrix}
\end{figure}